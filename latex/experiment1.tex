%TC:ignore
\documentclass[a4paper, 12pt]{article}

% \usepackage{import}
% \import{../}{requirements.tex}

% % \documentclass[14pt]{article}
% \documentclass{book}
% \documentclass[12pt, a4]{article}
\documentclass[12pt, a4]{book}
\linespread{1.5}
% Engine-specific settings
% Detect pdftex/xetex/luatex, and load appropriate font packages.
% This is inspired by the approach in the iftex package.
% pdftex:
\ifx\pdfmatch\undefined
\else
    \usepackage[T1]{fontenc}
    \usepackage[utf8]{inputenc}
\fi
% xetex:
\ifx\XeTeXinterchartoks\undefined
\else
    \usepackage{fontspec}
    \defaultfontfeatures{Ligatures=TeX}
\fi
% luatex:
\ifx\directlua\undefined
\else
    \usepackage{fontspec}
\fi
% End engine-specific settings

\usepackage{amsmath,amssymb}
\usepackage{fullpage}

\usepackage{graphicx,wrapfig,lipsum}
\usepackage[svgnames, dvipsnames]{xcolor}
\usepackage{url}
\urlstyle{same}

\usepackage[makestderr]{pythontex}
\restartpythontexsession{\thesection}


\usepackage[framemethod=TikZ]{mdframed}

\newcommand{\pytex}{Python\TeX}
% \renewcommand*{\thefootnote}{\fnsymbol{footnote}}

\usepackage[comma,authoryear,round]{natbib}
\usepackage{usebib}
\bibliographystyle{apsr}


% \usepackage{biblatex}
% \usepackage[
% backend=biber,
% style=alphabetic,
% sorting=ynt
% ]{biblatex}
% \addbibresource{MyLibrary.bib}
% \usepackage{harvard}
% \newbibfield{editor}
% \bibinput{\jobname}

% \bibliographystyle{abbrvnat}

% \setcitestyle{authoryear} %Citation-related commands
% \setcitestyle{citesep={;}, aysep={,}}


% \usepackage[dvipsnames]{xcolor}

\graphicspath{{images/}}
%
% \usepackage{mathptmx}      % use Times fonts if available on your TeX system
%
% insert here the call for the packages your document requires
\usepackage{latexsym}
\usepackage{hyperref} 
\usepackage{mathtools}
\usepackage{calculator}
\usepackage{longtable}
\usepackage{booktabs}

\usepackage{siunitx}
\usepackage{url}
\sisetup{round-mode=places,round-precision=0}

\usepackage{pgfplots}
\usepgfplotslibrary{polar}
\usepgflibrary{shapes.geometric}
\usetikzlibrary{calc}
\pgfplotsset{my style/.append style={axis x line=middle, axis y line=middle, xlabel={$x$}, ylabel={$y$}, axis equal }}

% graphicx latexsym hyperref mathtools amssymb siunitx url

% \smartqed  % flush right qed marks, e.g. at end of proof
\RequirePackage{fix-cm}
\usepackage{enumerate}
\usepackage{manfnt}
\usepackage{tikz-cd}
\usetikzlibrary{automata, positioning, arrows}
\usetikzlibrary{decorations.pathreplacing,positioning, arrows.meta}

\usepackage{tabularx,ragged2e}
\newcolumntype{C}{>{\Centering\arraybackslash\hspace{0pt}}X}
\newcolumntype{Y}{>{\RaggedRight\arraybackslash\hspace{0pt}}X}

\usepackage{color}
\hypersetup{
    linktoc=black, % 'all' will create links for everything in the TOC
    colorlinks=true,
    linkcolor=black,
    filecolor=magenta,
    urlcolor=black,
    pdftitle={Sharelatex Example},
    bookmarks=true,
    pdfpagemode=FullScreen,
    citecolor=black
}

\definecolor{aliceblue}{HTML}{00F9DE}

\usepackage{amsfonts}
\usepackage{bondgraphs}
\usepackage{xfrac}
\usepackage[normalem]{ulem}

% \usepackage{slashbox}
\usepackage{diagbox}

\usepackage{multicol}

\usepackage{ifthen}
\newboolean{firstanswerofthechapter}  


\colorlet{lightcyan}{cyan!40!white}

\usepackage{chngcntr}
\usepackage{stackengine}
\usepackage{rotating}

\usepackage{tasks}
\newlength{\longestlabel}
\settowidth{\longestlabel}{\bfseries viii.}
\settasks{label=\roman*., label-format={\bfseries}, label-width=\longestlabel,
item-indent=0pt, label-offset=2pt, column-sep={10pt}}

\usepackage[lastexercise,answerdelayed]{exercise}
\counterwithin{Exercise}{chapter}
\counterwithin{Answer}{chapter}
\counterwithin{Solution}{chapter}

\renewcounter{Exercise}[chapter]
\newcommand{\QuestionNB}{\bfseries\arabic{Question}.\ }

\renewcommand{\ExerciseName}{Puzzle}
\renewcommand{\ExerciseHeader}{\def\stackalignment{l}% code from https://tex.stackexchange.com/a/195118/101651
    \stackunder[0pt]{\colorbox{cyan}{\textcolor{white}{\textbf{\LARGE\ExerciseHeaderNB\;\large\ExerciseName}}}}{\textcolor{lightcyan}{\rule{\linewidth}{2pt}}}\medskip}

\newcounter{Puzzle}
\newenvironment{Puzzle}{\begin{Exercise}[name={Puzzle},
counter={Puzzle}]}
{\end{Exercise}}


\renewcommand{\AnswerName}{Exercises}
\renewcommand{\AnswerHeader}{\ifthenelse{\boolean{firstanswerofthechapter}}%
    {\bigskip\noindent\textcolor{cyan}{\textbf{CHAPTER \thechapter}}\newline\newline%
        \noindent\bfseries\emph{\textcolor{cyan}{\AnswerName\ \ExerciseHeaderNB, page %
                \pageref{\AnswerRef}}}\smallskip}
    {\noindent\bfseries\emph{\textcolor{cyan}{\AnswerName\ \ExerciseHeaderNB, page \pageref{\AnswerRef}}}\smallskip}}
\renewcommand{\AnswerName}{Exercises}
\renewcommand{\AnswerHeader}{\ifthenelse{\boolean{firstanswerofthechapter}}%
    {\bigskip\noindent\textcolor{cyan}{\textbf{CHAPTER \thechapter}}\newline\newline%
        \noindent\bfseries\emph{\textcolor{cyan}{\AnswerName\ \ExerciseHeaderNB, page %
                \pageref{\AnswerRef}}}\smallskip}
    {\noindent\bfseries\emph{\textcolor{cyan}{\AnswerName\ \ExerciseHeaderNB, page \pageref{\AnswerRef}}}\smallskip}} 
    
\setlength{\QuestionIndent}{16pt}

% \usepackage{caption}
\usepackage{float} % USED TO FORCE IMAGES INLINE
\usepackage{subcaption}
\usepackage{geometry}
\usepackage{pdflscape}
% \geometry{margin=1.8in}
% \usepackage[a4paper, total={6in, 8in}]{geometry}
% \geometry{margin=1.3in}
% \linespread{1.3}

\geometry{margin=1.4in}
\linespread{1.1}

\pagenumbering{gobble}
\usepackage{verbatim}
\immediate\write18{texcount -tex -sum  \jobname.tex > \jobname.wordcount.tex}

% Keywords command
\providecommand{\keywords}[1]
{
  \small	
  \textbf{\textit{Keywords---}} #1
}

% \usepackage{xcolor}     % for colour
% \usepackage{lipsum}     % for sample text
\usepackage{ntheorem}   % for theorem-like environments
% \usepackage{mdframed}   % for framing

\theoremstyle{break}
\theoremheaderfont{\bfseries}

\newmdtheoremenv[%
linecolor=gray,
leftmargin=60,%
rightmargin=60,
backgroundcolor=gray!40,%
innertopmargin=10,%
ntheorem]{puzzle}{Puzzle}[section]

\newmdtheoremenv[%
linecolor=gray,
leftmargin=60,%
rightmargin=60,
backgroundcolor=gray!7,%
innertopmargin=10,%
ntheorem]{solution}{Solution}[section]



% \newcommand{\mpp}{\textsc{\textbf{mpp}}}
% \newcommand{\mpps}{\textsc{\textbf{mpp }}}
\newcommand{\mpp}{MPP}
\newcommand{\mpps}{MPP }
\newcommand{\SE}{\textit{Stesibuque Engin}}
\newcommand{\SEs}{\textit{Stesibuque Engin} }

\newcommand{\bce}{\textsc{bce}}
\newcommand{\bces}{\textsc{bce }}
\newcommand{\CE}{\textsc{ce}}
\newcommand{\CEs}{\textsc{ce }}
\newcommand{\ImageWidth}{11cm}

% Geometric scaling
\newcommand{\dualsplit}{.35\linewidth}
\newcommand{\dualadj}{\dualsplit/12}

\newcommand{\tripplesplit}{.3\linewidth}
\newcommand{\triadj}{\tripplesplit/12}

\usepackage{textcomp}

\usepackage{cleveref}
\crefname{section}{§}{§§}
\Crefname{section}{§}{§§}

\pagenumbering{arabic}




\title{Reinventing the (Water) Wheel \\
\\
\\
\small{Experimental design #1}}
\author{Sholto Maud}
% \keywords{Experimental Philosophy, Principle Based Design}
\date{\today}

%TC:endignore


% come up with x6 key terms use 2-3 of them in your title (see note about 3D printing) Consider discoverability for your title & search engine optimisation

\pagenumbering{arabic}

\begin{document}
\makeindex
\maketitle
\tableofcontents
% \listoffigures
% \listoftables





% The first aim of the thesis is to show the \mpps in operation instrumentally through a `research as practice' design research methodology.

% 5223_RINT1000
% Module 3 - Planning Your Research
% Module 3

% Aims
% What do you intend to do?
% What are your aims or objectives?
% Have you developed your hypothesis?

% Prior work
% What prior research has been done on your topic?
% What are the strengths and shortcomings of prior research?

% Justify
% Why should this research be undertaken?
% Will it advance knowledge/understanding?
% Will it have useful outcomes?

% Data management
% Who owns the data/information you are collecting?
% How will you collect and manage the data/information you collect?
% How will you store the data/information you collect?

% Expected outcomes
% What are your anticipated findings or outcomes?
% How will you know your findings are valid?
% Do you anticipate any problems or shortcomings?
% Are publications an expected outcome?

% Budget
% What will it cost to complete your project?
% Could you do part of the research if full funding is not available?
% Are all of the expenses justifiable?

% Contributors
% Who will be working on the project?
% Are they participating as team members or collaborators?
% What experience do they have?
% Have they been properly trained?
% Who will supervise them?

% Dissemination
% How should your findings best be communicated?
% How will you logically order and group your findings for dissemination?
% Will you present your findings? Where will you present them (e.g. at conferences, in peer-reviewed journals, as a book)?
% Do you need to communicate your findings to a stakeholder or sponsor?
% Do you need to communicate your findings to any study participants?
% Have you considered who qualifies as authors for each presentation or publication?
% Are there contributors you need to acknowledge?

\newpage


\section{Introduction}

\begin{figure}[ht!]
    \centering
    \includegraphics[width=0.9\linewidth]{ewbank_descriptive_1857_book1_p2.jpg}
    \caption{The scholar mourns \cite[p.~2]{ewbank_descriptive_1857}}
    \label{fig:ewbank}
\end{figure}

The extract shown in Figure~\ref{fig:ewbank} is taken from a book on ancient and modern machines for raising water by the former United States Commissioner of Patents (1849-1852) Thomas Ewbank.\footnote{For a brief account of \citeauthor{ewbank_descriptive_1857}'s time as Commissioner of Patents see \cite{bate_thomas_1973}.} The extract is given here to show that even as late as \citeyear{ewbank_descriptive_1857} some of the most qualified technical experts believed, not only that principles of science had informed the design and construction of hydraulic machines, but also that the description of such principles had not been preserved in either the ancient or early modern literature.

The present set of experiments attempt to show such principles, and focus on the idea that a principle can be used to inform design. The intent is to unveil at least one such principle using the undershot vertical waterwheel as a design case. Extracts from the writing of various scholars from bygone eras---specifically \citeauthor{descartes_principles_1982}, \citeauthor{smeaton_experimental_1759} and \citeauthor{ewbank_descriptive_1857}---will be provided here to introduce some context to these experiments.



\citeauthor{ewbank_descriptive_1857} attempted to explain the apparent absence of principles in the literature by referring to what he claimed was a convention of philosophers and scholarship at the time:

\begin{figure}[ht!]
    \centering
    \includegraphics[width=0.9\linewidth]{ewbank_descriptive_1857_vulgar_p2.jpg}
    \caption{The vulgar classes \cite[p.~2]{ewbank_descriptive_1857}}
    \label{fig:ewbank:vulgar}
\end{figure}

Whilst there might be some truth to \citeauthor{ewbank_descriptive_1857}'s claim, there is evidence of an early modern tradition that was concerned with the derivation of principles. That tradition became known as Cartesianism, and followed \citeauthor{descartes_principles_1982}'s key work, \textit{The Principles of Philosophy} \citep{descartes_principles_1982}. The history of scholarship that followed \citeauthor{descartes_principles_1982}'s publications became embroiled in many controversies which appear to have made it difficult to establish indubitable facts. But what we know of the period is that key thinkers such as Newton and Leibniz had engaged deeply with \citeauthor{descartes_principles_1982}'s writing. It is, then, reasonable to assume that \citeauthor{descartes_principles_1982}'s insistence about the relevancy of principles may have stimulated subsequent design thinking. In addition, following along with \citeauthor{ewbank_descriptive_1857}'s argument, it is possible that \citeauthor{descartes_principles_1982}'s principles had been directly applied by designers but without attribution or acknowledgement.

The eighteenth century publications on undershot water wheel design and analysis by \citeauthor{smeaton_experimental_1759}, one of the first professional Civil Engineers, partly qualify as examples of \citeauthor{ewbank_descriptive_1857}'s argument. I say `partly' here because \citeauthor{smeaton_experimental_1759}'s writing does in fact mention the idea of a principle, but does not mention its origin, nor how to derive a principle. One principle mentioned by \citeauthor{smeaton_experimental_1759} will be the starting point for the current experiments. It is a principle whose constancy is between motion and geometry.




, But despite this argument it is possible to see mention of principles in some early engineering literature in the .




Regarding the


In \citeyear{smeaton_experimental_1759} \citeauthor{smeaton_experimental_1759} published a ground-breaking experimental analysis of the undershot water wheel. In that analysis \citeyear{smeaton_experimental_1759}


\section{Experiment 1.}

\subsection{Aim}

The series of experiments outlined here aim to test and verify the two principles identified in the introductory section which state that to optimise the effectiveness of an ancient waterwheel the wheel circumference should equal the stream velocity.

\subsection{Method}

Following the V-Modell outlined in the methodology section, each experiment will consist of three parts. A digital simulation and a physical experiment as the test case. In this case, for the simulation, basic methods of geometry and arithmetic will be used to present HTML5 simulation of \citeyear{smeaton_experimental_1759}'s concept, displayed on \cite{maud_principle_2022}. The basic method will then be transferred from HTML5 into a CAD representation using Rhino3D/Grasshopper, and the CAM will be a 3D fabrication. The physical experiment will then be conducted using a basic stop watch over a distance.

\subsection{Discussion}

Although the simulation experiment The aim as presented above provided two difficulties.


The difficulty becomes evident through an initial dimensional analysis.

If we only focus on circumference and velocity (distance over time) then we have no






B. the ratio of oblique to perpendicular force will equal the ratio between the sine of  the angle of incidence (oblique angle) and (sine of?) 90 degrees.



\section{Method}


\bibliography{MyLibrary}

\printbibliography

\end{document}
